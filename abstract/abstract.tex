
\addcontentsline{toc}{chapter}{Abstract}

\begin{abstract}

It is known from aeromagnetic surveys from Venezuela, China and the U.S.A. that it is common to find magnetic anomalies over oil fields. It has been suggested that hydrocarbon seepage creates physical and geochemical conditions in soils and rocks that are conducive to \textit{in situ} formation of secondary ferromagnetic minerals such as magnetite and greigite. In near-surface samples drilled in Venezuela it has been observed that these minerals form aggregates of roughly spherical grains called \textit{framboids}. While the specific physical and geochemical processes involved in the genesis of these minerals is an active research topic, I will focus in studying the potentially unique magnetic signature of these minerals via numerical simulation of the \textit{micromagnetic} equations. Micromagnetics is the theoretical framework that bridges the fundamental quantum mechanical description of ferromagnetism and its effective macroscopic description by Maxwell's equations. Solving the micromagnetic problem can be done either by minimising the total magnetic Gibbs free energy or by solving the continuum-limit dynamical equation, that is, the Landau-Lifshitz-Gilbert (LLG) equation. I will use a finite element method based on the general-purpose package collection for automated solutions of partial differential equations \textit{FEniCS} to solve the LLG equation as well as a conjugate gradient method for energy minimisation. My aim is to find the magnetic response of framboidal aggregates to standard rock-magnetic and palaeomagnetic techniques. Most previous simulations have been performed by finite difference methods and thus have been limited to somewhat unrealistic shapes like cubes and rectangular prisms. As a first approach to the problem, before scaling up the models, I have conducted simulations of hysteresis loops and zero-field domain structure of octahedral-shaped single grains of greigite and compared these with octahedral grains whose corners have been removed to more accurately study the morphologies seen in nature.

\end{abstract}